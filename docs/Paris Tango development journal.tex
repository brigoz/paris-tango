% Created 2019-03-21 Thu 08:47
% Intended LaTeX compiler: pdflatex
\documentclass{article}
               \usepackage{hyperref}
               \usepackage{fontspec}
               \usepackage{graphicx}
               \usepackage{longtable}
               \defaultfontfeatures{Mapping=tex-text}
               \setromanfont{Calluna Light}
               \setromanfont [BoldFont={Calluna Bold},
                               ItalicFont={Calluna Italic}]{Calluna Light}
               \setsansfont{Calluna Sans}
               \setmonofont[Scale=0.98]{Anonymous Pro}
               \usepackage{geometry}
               \geometry{a4paper, textwidth=6.5in, textheight=10in,
                                   marginparsep=7pt, marginparwidth=.6in}
               \pagestyle{empty}
               \linespread{1.135}
               \title{}
\author{Sacha El Masry}
\date{\today}
\title{Paris Tango: Development Journal}
\hypersetup{
 pdfauthor={Sacha El Masry},
 pdftitle={Paris Tango: Development Journal},
 pdfkeywords={},
 pdfsubject={},
 pdfcreator={Emacs 25.3.1 (Org mode 9.1.6)}, 
 pdflang={English}}
\begin{document}

\maketitle
\tableofcontents


\section{Project Summary}
\label{sec:org1728773}
This project sets out to modernise the current, static, Paris Tango website,
originally designed by me in 2008. Since then many things have changed: desktop
programs allowing visual editing of the static code have been dropped and are no
longer usable. Smartphones and tablets have become the norm, and now demand a
new level of targeted development to ensure not only that the website displays
correctly to mobile users, but also that it is appealing to use, and that
visitors use it for its inteded purpose, finding out about tango events and
booking them. At the same time, and no less driven by mobile devices and their
variable connection speeds to the internet, images and videos now have to be
given special treatment, optimising each media asset to load as quickly as
possible on a visitor's device. Also, there has been a complete shift on the
internet to encrypted communications, so that contact forms can't be \emph{listened
to} as they are submitted by a visitor and sent to the event organiser. To do
this, all modern websites now need a TLS (formerly known as SSL) certificate.

All these changes, and a few besides, are now carefully watched by Google, whose
directory will ruthlessly down-rank any website not matching all these criteria,
which is ultimately disastrous for new business, as it makes it less likely that
Paris Tango will be discovered by a random search.

The aim of this project is to make it easy for the site's administrator to
easily update content, without needing to hire a web designer every time she
needs to, to automate asset optimisation, to make it effortless to obtain and
renew TLS certificates, and to remove any further need to worry about website
hosting, concentrating only on the content.

\section{Preparation}
\label{sec:org9fe5558}

\subsection{Requirements-gathering}
\label{sec:orgf860396}

\subsection{Collecting necessary technical information}
\label{sec:orga63dbe8}

\subsection{Obtaining all assets}
\label{sec:org7ea624a}
To build a completely new website, we need to use a completely new set of
images; in the process of first developing the site, it was enough to get images
(photos, newspaper and magazine scans) with a resolution of up to 1024 pixels,
as all visitors would browse the site from their home desktop or mobile laptop,
using Internet Explorer, Firefox, Safari and Opera as browsers.

Since then, iPhones and all the other smartphones became popular, and their use
now dwarfs the number of visits from desktop and laptop computers, as well as
tablets, which also didn't exist at that time. All images now have to be made
\emph{responsive}, so that each device and browser only download the quality of image
that they need. For example, a low-end smartphone on a slow 3G connection will
download a low resolution version of an image, while a high-end tablet on a 4G
or WiFi connection will download a very high resolution image, for the site to
present well. This makes it necessary to start with very high resolution imagery
(1600 pixels and above), \emph{and} to provide that same image in multiple
resolutions to satisfy a very wide range of devices which people will be using
to visit the site.

With this limitation, it is no longer possible to use images from the first
site; even in that period it was important to select a target resolution and to
optimise the image. In the process of optimisation and resizing, the image is
further compressed, losing detail in the process. So, for any images which are
to be reused, it is important to go back to the highest-quality original and
prepare it for use, again, or to select a better, newer and higher-resolution
image instead.

Finally, one of the design decisions was to make the new site more graphical and
less text-heavy. With that in mind, Brigitte and I---as well as Conny---went
through all photos accumulated over the intervening years and events to find a
new set of all images that can be used both in generic, non-specific settings to
add atmosphere to the website, as well as images of specific events whose
presence is going to be kept up to provide \emph{history} and artistic merit.

\subsection{Creation of all necessary service provider accounts}
\label{sec:org4752663}

\section{Building the Site}
\label{sec:org4ea48d8}

\subsection{Preparation of new website repository}
\label{sec:org52aff36}

\subsection{Preparation of special content types}
\label{sec:org931c2bd}
An events type of page is necessary to deal with organising and presenting a
growing list of tango events. This requires the definition of a specific
\emph{archetype} as well as \emph{layouts} or \emph{views} for presenting the events as
individual pages or a listing.

\subsection{Migration of existing content to new platform}
\label{sec:orgc376395}
In this stage, all the content from the legacy site is copied from the existing,
legacy website to the new Hugo-based one. All pages are recreated, and any
\emph{custom} content, events in our case, are modelled and built as \emph{fields} in the
site generator.

\subsection{Asset optimisation}
\label{sec:orgcf41f84}
All imagery used is uploaded onto the Cloudinary server, where Forestry.io can
find it to use it in the content management system. But, it isn't enough to
simply upload all the images as they are; many photos are huge, which is useful
to get the best quality out of them, but it's very wasteful, too. As a
preliminary step, all photos will be resized to a maximum width of 1920 pixels,
then optimised heavily, before uploading to Cloudinary.

The images are optimised by creating a copy of the containing folder, after
which all images are opened in macOS Preview, and resized to a maximum of 1920
pixels width.


\subsection{Crafting a new look for the web site}
\label{sec:org30718be}

\subsubsection{Selecting and installing a theme}
\label{sec:orgfe0b241}
To keep development costs down, instead of designing a fully custom them, we
will make use of one of the existing, provided, \emph{\href{https://themes.gohugo.io/aether/}{themes}} for Hugo: \href{https://themes.gohugo.io/forty/}{Forty}, and customise
that to a limited extent. Forty was chosen out of a possible further choice of
\href{https://themes.gohugo.io/hugo-tracks-theme/}{Tracks} and \href{https://themes.gohugo.io/aether/}{Aether}. Aether is a very elegant and beautiful theme, but it is
primarily targeted at blogs, which may later make it harder to customise for
general purpose websites. While Tracks is also very good, Forty just seems to be
both more recently maintained, and more generally flexible for generic website
use.

\subsubsection{Customising the base theme}
\label{sec:orga0df261}
Forty, the theme we're basing the entire look of the new site on, at least in
early 2019, carries its own developer's opinions which show up as aesthetic
choices across the theme. While this theme is elegant and looks good, it's not
desirable to leave it as is; Paris Tango has its own established look and feel,
and it will not benefit the business to use the theme and look like countless
other websites also using it. Thus, the theme needs to be customised, at least
superficially, to give it more of the previous, but modernised, \emph{Paris Tango}
look.

To start with, the biggest graphical element on the site is the image of the
\emph{burning} orange curtains. Forty expects this image to be called \texttt{banner.jpg}
and placed in the \texttt{static/img} folder. As whatever image is placed in the top
banner is likely to have distracting elements, vivid colours and many changes of
contrast which will make the overlay text difficult to read, an overlay is used
to dampen the image, to reduce its contrasts. Forty comes pre-built with a
blue-tinted overlay, which has to change to better hew to the orange hues of the
website. My starting point is to use a more complementary colour for the
overlay, to fit the orange theme, and to make the background image more vivid
while at the same time reducing its contrast. To this end, I've chosen the
\href{https://www.pantone.com/color-intelligence/color-of-the-year/color-of-the-year-2019}{Pantone colour of the year 2019}, \emph{\href{https://www.pantone.com/color-finder/16-1546-TPX}{Living Coral (16-1546)}}, with a peachy-orange
hue (RGB: 250, 114, 104 | Hex: FA7268), with an opacity of 0.5 to hide certain
dull details present in the image. The rest of the monochrome colour scheme is
in this palette: \url{https://coolors.co/820f07-a01c13-bf3026-fa7268-ff9189}.

\section{Deploying the New Website}
\label{sec:org6988b37}
\subsection{Change DNS records}
\label{sec:orga46cdb7}
The domain name is registered with \href{https://godaddy.com/}{GoDaddy}, using their default nameservers:

\begin{enumerate}
\item ns09.domaincontrol.com
\item ns10.domaincontrol.com
\end{enumerate}

The following are all the DNS entries recorded for paris-tango.co.uk:

\begin{center}
\begin{tabular}{llll}
Type & Name & Value & TTL\\
\hline
A & @ & 160.153.129.215 & 10800\\
A & admin & 160.153.129.215 & 10800\\
A & mail & 160.153.129.215 & 10800\\
CNAME & autodiscover & autodiscover.outlook.com & 1 hour\\
CNAME & cpanel & @ & 10800\\
CNAME & selector1.\textsubscript{domainkey.paris}-tango.co.uk & selector1-paris-tango-co-uk.\textsubscript{domainkey.paristango.onmicrosoft.com} & 1 hour\\
CNAME & selector2.\textsubscript{domainkey.paris}-tango.co.uk & selector2-paris-tango-co-uk.\textsubscript{domainkey.paristango.onmicrosoft.com} & 1 hour\\
CNAME & webdisk & @ & 10800\\
CNAME & webdisk.admin & @ & 10800\\
CNAME & whm & @ & 10800\\
CNAME & www & @ & 10800\\
CNAME & www.admin & @ & 10800\\
CNAME & \_domainconnect & \_domainconnect.gd.domaincontrol.com & 1 hour\\
MX & @ & paristango-co-uk01c.mail.protection.outlook.com & 1 hour\\
NS & @ & ns09.domaincontrol.com & 1 hour\\
NS & @ & ns10.domaincontrol.com & 1 hour\\
SOA & @ & Primary nameserver: ns09.domaincontrol.com & 1 hour\\
TXT & @ & v=spf1 include:spf.protection.outlook.com -all & 1 hour\\
TXT & selector1.\textsubscript{domainkey} & selector1-paris-tango-co-uk.\textsubscript{domainkey.paristango.onmicrosoft.com} & 1 hour\\
\end{tabular}
\end{center}


\subsection{Implement final corrections}
\label{sec:orgbaeebc4}
The photographs I selected to put in their respective galleries aren't
correct, adequate or doing the job. Brigitte has re-sent what appears to be
the same set of images she already provided throughout the project, but
sorted into albums this time, making it easy to download, manipulate them for
maximum size, optimise them, upload them to Cloudinary and finally to insert
them into their respective album.

Updated galleries, in a live telephone conference with Brigitte, moving
contents out of agency into events and festivals, moving certain imagery from
friends to agency, and adding images to all three as per instructions.

Change colour of logo backing box to be the same darker shade of the primary
colour as used in the footer of the page, in the social media icons section.
Also, add a little bit of sideways margin to let the logo \emph{breathe} a little.

Update templated social icons in page footer: ensure that the social icon
\emph{label} text appears next to the icon. By default, only the icon shows, but
the image used in the icon can be ambiguous, so additionally write i.e. ``Facebook
page'' next to the icon.

Out of the box, the template comes with a single brickwork-like section to
display a showcase. As we have decided to use it multiply, for featured pages
and photo galleries, it started to lose visual distinction. To add a
configurable section between needed changes in the \texttt{config.toml} file, as
well as the \texttt{tile.html} layout file. It is now possible to enable or disable
the entire photo gallery section, or just the separator with the title and
subtitle. Also, the change meant that the tile section repeated layout, with
the narrow tile on the left, followed by a wide (60\%) tile, followed in the
second line by a wide then narrow tile again. This was visually jarring,
ruining visual consistency, leading to a specification of styles for an
alternate tile section, with class \texttt{tile-{}-alternate} specifying which layout
the section is to take.


\section{Time Log}
\label{sec:org0c5bcad}

\begin{table}[htbp]
\caption{Clock summary at \textit{[2019-03-21 Thu 01:59]}}
\centering
\begin{tabular}{lrrl}
Headline & Time &  & \\
\hline
\textbf{Total time} & \textbf{34:25} &  & \\
\hline
Project Summary & 1:01 &  & \\
Preparation & 7:15 &  & \\
\hspace*{1.0em}Requirements-gathering &  & 2:00 & \\
\hspace*{1.0em}Collecting necessary technical\ldots{} &  & 1:10 & \\
\hspace*{1.0em}Obtaining all assets &  & 2:19 & \\
\hspace*{1.0em}Creation of all necessary service\ldots{} &  & 1:46 & \\
Building the Site & 19:00 &  & \\
\hspace*{1.0em}Preparation of new website repository &  & 1:26 & \\
\hspace*{1.0em}Preparation of special content types &  & 1:59 & \\
\hspace*{1.0em}Migration of existing content to new\ldots{} &  & 4:38 & \\
\hspace*{1.0em}Asset optimisation &  & 2:25 & \\
\hspace*{1.0em}Crafting a new look for the web site &  & 8:32 & \\
\hspace*{2.0em}Selecting and installing a theme &  &  & 2:50\\
\hspace*{2.0em}Customising the base theme &  &  & 5:42\\
Deploying the New Website & 7:09 &  & \\
\hspace*{1.0em}Change DNS records &  & 1:05 & \\
\hspace*{1.0em}Implement final corrections &  & 4:25 & \\
\end{tabular}
\end{table}
\end{document}
